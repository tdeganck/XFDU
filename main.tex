\documentclass[12pt,a4paper]{article} % classe article, taille 12, papier A4
\usepackage{fontspec} % permet de gérer la police

\usepackage{graphicx} % gère les images
\usepackage{array} % gère les tableaux
\usepackage{hyperref} % gère les liens
\usepackage{csquotes} % gère les citations et les guillemets

\usepackage{polyglossia} % gère l'aspect multilingue des documents
\setmainlanguage{french} % sélection de la langue principale du document

\usepackage[citestyle=verbose]{biblatex} % appel de biblatex, de nombreuses autres options disponibles
\bibliography{nom_du_fichier_bib} % appel du fichier .bib sans l'extension

% emplacement pour d'autres packages

\title{L'empaquetage des informations pour leur préservation à long terme} % le titre du document
\author{Tommy De Ganck} % son auteur
\date{8 janvier 2024} % la date


%% début du document

\begin{document} % début du corps du texte

\maketitle % affiche le titre, l'auteur et la date
	

\section{Introduction} % présentation de la pertinence, de l'objectif et de la structure de l'article
Le présent travail s'intéresse aux formats d'empaquetage des informations pour leur préservation à long terme. Dans ce contexte, l'empaquetage consiste à conditionner le contenu à archiver avec les métadonnées qui s'y rapportent. Pour qu'un tel paquet soit identifiable, lisible et donc utilisable, il est nécessaire de décrire la structure du paquet ainsi que les liens entre le contenu archivé (et ses différents éléments), d'une part, et les métadonnées qui leur sont associées, d'autre part. 

L'empaquetage est un élément clé de tout projet d'archivage électronique. En effet, la conservation pérenne des contenus implique de les documenter suffisamment afin d'être capable d'en assurer une gestion optimale dans le temps, d'y effectuer des recherches et d'en permettre la consultation (par virtualisation notamment). Or, les contenus à conserver sont non seulement de plus en plus nombreux, mais également de plus en plus complexes. L'empaquetage des informations à archiver doit donc répondre au double défi d'un encapsulage performant (qui décrit et relie tous les éléments utiles d'une façon structurée) et économique (qui évite les redondances sans compromettre la pérennité des l'accès aux ressources nécessaires). 

Cette question est abordée ici dans le contexte de l'\textit{Open Archival Information System} (OAIS), que l'on peut traduire par "Système ouvert d'archivage d'information". Deux standards d'empaquetage basé sur l'\textit{Extensible Markup Language} (XML) sont présentés et comparés : le \textit{Metadata Encoding and Transmission Standard} (METS) et le \textit{XML Formatted Data Unit} (XFDU). 

Après un bref historique de l'OAIS et des initiatives prises sur le plan international pour organiser son implémentation, j'introduis de façon synthétique le système décrit dans l'OAIS et plus spécifiquement la notion de "paquet d'information". Je présente ensuite successivement les standards METS et XFDU avant d'en esquisser une comparaison en guise de conclusion. 

\section{Contexte} % explication du contexte historique de l'OAIS et des initiatives liées à l'implémentation sur le plan international
Le cadre méthodologique général actuellement reconnu et utilisé internationalement pour l'archivage et à la préservation à long terme de documents numériques est l'\textit{Open Archival Information System} (OAIS). OAIS est un modèle conceptuel destiné à la gestion, à . Il a été élaboré par le \textit{Consultative Committee for Space Data Systems} (CCSDS), un organisme regroupant les principales agences spatiales du monde. Il a été publié pour la première fois en 2002 comme standard CCSDS. Le standard est rapidement enregistré entant que norme ISO (en 2003\footnote{sous la référence 14721:2003}). La norme ISO a fait l'objet d'une révision en 2012\footnote{sous la référence 14721:2012}.

L'OAIS définit les principes, les concepts, les responsabilités et les fonctions d'un système d'archivage numérique capable de préserver l'information sur le long terme. Les auteurs de la norme insistent sur le fait qu'il n'est pas possible de garantir une conservation éternelle, même si le but est de viser une préservation aussi longue que possible. Il faut dès lors comprendre le "long terme" comme une durée suffisamment longue que pour être soumis à l'impact des évolutions technologiques et des besoins des utilisateurs. OAIS est un modèle conceptuel général et ne fournit pas de spécifications techniques. Le modèle OAIS a plutôt pour vocation de fournir des lignes directrices et des bonnes pratiques pour concevoir et mettre en œuvre un système d'archivage numérique conforme aux exigences de pérennisation. Il propose un vocabulaire commun et un cadre théorique pour faciliter les échanges entre les différents acteurs impliqués\footnote{Les informations d'ordre général sur le modèle OAIS sont aisément et librement accessibles en ligne. Vous pouvez consulter \href{https://www.fr.wikipedia.org/wiki/Open_Archival_Information_System}{la fiche de Wikipedia dédiée à OAIS} et \href{http://www.oais.info/}{site d'information officiel sur le modèle}}.

Le cadre théorique du modèle OAIS se compose essentiellement de deux types de modèles tous les deux détaillés au sein de la norme : le modèle d'information et le modèle fonctionnel. Le modèle d'information décrit la structure et le contenu des informations archivées et créée à cette occasion une nomenclature spécifique (j'y reviens au point suivant). Il définit également les métadonnées associées à ces objets, qui permettent de les identifier, de les décrire, de les gérer et de les rendre accessibles. Le modèle fonctionnel décrit les activités et les services que doit assurer un système d'archivage numérique, regroupés en six fonctions principales : ingestion, gestion des données archivistiques, préservation, gestion de l'administration, gestion du planning et accès.

Largement accepté et utilisé  par diverses organisations et disciplines, tant nationales qu'internationales, OAIS constitue aujourd'hui une référence incontournable pour l'archivage numérique, mais reste insuffisante pour assurer sa mise en oeuvre. C'est pourquoi le modèle OAIS a servi de base à l'élaboration d'autres normes et standards associés qui sont autant de briques supplémentaires indispensables pour implémenter ses principes dans un projet concret d'archivage électronique. Ces normes et standards sont d'ailleurs renseignés en tant que tel au sein de la norme\footnote{Voyez \href{https://public.ccsds.org/Pubs/650x0m2(F).pdf}{les pratiques recommandées}} et une liste de ceux-ci est régulièrement mise à jour sur le site d'information officiel au sein d'un \href{http://www.oais.info/standards-process/oais-roadmap-and-related-standards/}{"parcours"} qui les présente dans l'ordre utile au niveau de la mise en oeuvre (depuis la préparation des données à ingérer jusqu'à la certification du système d'archivage mis en place).Une série d'étape au sein de ce parcours n'ont pas encore de standard spécifique disponible, ce qui met en lumière que l'effort de normalisation ne couvre pas encore tous les aspects de l'implémentation.

Logiquement, les normes et standards déjà mis au point concernent les aspects liés aux premières étapes de la mise en oeuvre : la préparation et l'ingestion des données et métadonnées. 


\section{Les paquets d'informations dans l'OAIS}

\section{Formats d'empaquetage}

\subsection{METS}

\subsection{XFDU}

\section{Conclusion}


%% bibliographie
\printbibliography

\tableofcontents % affiche une table des matières correspondant aux sections du document

\end{document}